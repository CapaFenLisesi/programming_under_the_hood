\chapter{Table of ASCII Codes}
\label{asciilisting}

% 
% 
% Copyright 2002 Jonathan Bartlett
% 
% Permission is granted to copy, distribute and/or modify this
% document under the terms of the GNU Free Documentation License,
% Version 1.1 or any later version published by the Free Software
% Foundation; with no Invariant Sections, with no Front-Cover Texts,
% and with no Back-Cover Texts.  A copy of the license is included in fdl.xml
% 

To use this table, simply find the character or escape that you want the
code for, and add the number on the left and the top.

\begin{table}[h]
\begin{tabular}{FIXME}

 & +0 & +1 & +2 & +3 & +4 & +5 & +6 & +7 & 
0 & NUL & SOH & STX & ETX & EOT & ENQ & ACK & BEL & 
8 & BS & HT & LF & VT & FF & CR & SO & SI & 
16 & DLE & DC1 & DC2 & DC3 & DC4 & NAK & SYN & ETB & 
24 & CAN & EM & SUB & ESC & FS & GS & RS & US & 
32 &  & ! & " & \# & \$ & \% & & & ' & 
40 & ( & ) & * & + & , & - & . & / & 
48 & 0 & 1 & 2 & 3 & 4 & 5 & 6 & 7 & 
56 & 8 & 9 & : & ; & < & = & > & ? & 
64 & @ & A & B & C & D & E & F & G & 
72 & H & I & J & K & L & M & N & O & 
80 & P & Q & R & S & T & U & V & W & 
88 & X & Y & Z & [ & \\ & ] & \textasciicircum & \_ & 
96 & ` & a & b & c & d & e & f & g & 
104 & h & i & j & k & l & m & n & o & 
112 & p & q & r & s & t & u & v & w & 
120 & x & y & z & { & | & } & ~ & DEL & 
\end{tabular}
\caption{Table of ASCII codes in decimal}
\end{table}

ASCII is actually being phased out in favor of an international standard
known as Unicode, which allows you to display any character from any known
writing system in the world.  As you may have noticed, ASCII only has support
for English characters.  Unicode is much more complicated, however, because
it requires more than one byte to encode a single character.  There are 
several different methods for encoding Unicode characters.  The most common
is UTF-8 and UTF-32.  UTF-8 is somewhat backwards-compatible with ASCII
(it is stored the same for English characters, but expands into multiple
byte for international characters).  UTF-32 simply requires four bytes for
each character rather than one.  Windows\textregistered uses UTF-16, which is a 
variable-length encoding which requires at least 2 bytes per character, so
it is not backwards-compatible with ASCII.

A good tutorial on internationalization issues, fonts, and Unicode is available
in a great Article by Joe Spolsky, called "The Absolute Minimum Every Software
Developer Absolutely, Positively Must Know About Unicode and Character Sets
(No Excuses!)", available online at http://www.joelonsoftware.com/articles/Unicode.html

