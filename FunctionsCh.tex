\chapter{All About Functions}
\label{functionschapter}

% 
% 
% Copyright 2002 Jonathan Bartlett
% 
% Permission is granted to copy, distribute and/or modify this
% document under the terms of the GNU Free Documentation License,
% Version 1.1 or any later version published by the Free Software
% Foundation; with no Invariant Sections, with no Front-Cover Texts,
% and with no Back-Cover Texts.  A copy of the license is included in fdl.xml
% 

\section{Dealing with Complexity}

In \autoref{firstprogs}, the programs we wrote only consisted
of one section of code.  However, if we wrote real programs like that,
it would be impossible to maintain them.  It would be really difficult
to get multiple people working on the project, as any change in one part 
might adversely affect another part that another developer is working on.

To assist programmers in working together in groups, it is necessary
to break programs apart into separate pieces, which communicate with
each other through well-defined interfaces.  This way, each piece can
be developed and tested independently of the others, making it easier
for multiple programmers to work on the project.

Programmers use \emph{functions\index{functions}} to break their programs
into pieces which can be independently developed and tested.  Functions
are units of code that do a defined piece of work on specified types of 
data.  For example, in a word processor program, I may have a function called
\icode{handle\_typed\_character} which is activated whenever a
user types in a key.  The data the function uses would probably be the
keypress itself and the document the user currently has open.  The 
function would then modify the document according to the keypress it was 
told about.

The data items a function is given to process are called its 
\emph{parameters\index{parameters}}.  In the word processing example, the
key which was pressed and the document would be considered parameters 
to the \icode{handle\_typed\_characters} function.  The parameter
list and the processing expectations of a function (what it is expected to do
with the parameters) are called the function's interface.  Much care
goes into designing function interfaces, because if they
are called from many places within a project, it is difficult to change
them if necessary.

A typical program is composed of hundreds or thousands of functions, each with a 
small, well-defined task to perform.  However, ultimately there are things
that you cannot write functions for which must be provided by the system.
Those are called \emph{primitive functions\index{primitive functions}} (or just \emph{primitives\index{primitives}}) - they are the
basics which everything else is built off of.  For example, imagine a program
that draws a graphical user interface.  There has to be a function to create
the menus.  That function probably calls other functions to write text, to
write icons, to paint the background, calculate where the mouse pointer is,
etc.  However, ultimately, they will
reach a set of primitives provided by the operating system to do basic line
or point drawing.  Programming can either be viewed as breaking a large
program down into smaller pieces until you get to the primitive functions,
or incrementally building functions on top of primitives until you get the large picture
in focus.  In assembly language, the primitives are usually the same thing
as the system calls\index{system calls},
even though system calls aren't true functions as we will talk about in this chapter.

\section{How Functions Work}
\label{howfunctionswork}

Functions are composed of several different pieces:

\begin{description}
\item[function name] A function's name is a symbol\index{symbol}
that represents the address where the 
function's code starts.  In assembly language, the symbol is defined
by typing the function's name as a label
before the function's code.  This is just like labels\index{labels} you have used
for jumping.
\item[function parameters] A function's parameters\index{parameters}
are the data items that are explicitly
given to the function for processing.  For example, in mathematics,
there is a sine function.  If you were to ask a computer to find the sine
of 2, sine would be the function's name, and 2 would be the parameter.  Some
functions have many parameters, others have none.\footnote{Function parameters
can also be used to hold pointers to data that the function wants to send back to the
program.}
\item[local variables] Local variables\index{local variables} 
are data storage that a function uses
while processing that is thrown away when it returns.  It's kind of like
a scratch pad of paper.  Functions get a new piece of paper every time they
are activated, and they have to throw it away when they are
finished processing.  Local variables of a function are not accessible
to any other function within a program.
\item[static variables] Static variables\index{static variables} are data storage that a function
uses while processing that is not thrown away afterwards, but is
reused for every time the function's code is activated.  This data
is not accessible to any other part of the program.  Static
variables are generally  not used unless absolutely necessary, as they
can cause problems later on.
\item[global variables] Global variables
are data storage that a function uses for processing
which are managed outside the function.  For example, a simple text editor
may put the entire contents of the file it is working on in a global
variable so it doesn't have to be passed to every function that operates
on it.\footnote{This is generally considered bad practice.  Imagine
if a program is written this way, and in the next version they decided
to allow a single instance of the program edit multiple files.  Each function
would then have to be modified so that the file that was being manipulated
would be passed as a parameter.  If you had simply passed it as a parameter
to begin with, most of your functions could have survived your upgrade
unchanged.}  Configuration values are also often stored in
global variables.
\item[return address] The return address\index{return address} 
is an "invisible" parameter in that it isn't directly used during the function.
The return address is a parameter which tells the function where to resume
executing after the function is completed.   This is needed because 
functions can be
called to do processing from many different parts of your program, and
the function needs to be able to get back to wherever it was called
from.  In most programming languages, this parameter
is passed automatically when the function is called.  In assembly language,
the \icode{call\index{call}} instruction handles passing
the return address for you, and \icode{ret\index{ret}} handles
using that address to return back to where you called the function from.
\item[return value] The return value\index{return value} is the 
main method of transferring data back to the
main program.  Most programming languages only allow a single return value 
for a function.
\end{description}

\index{global variables} 
These pieces are present in most programming languages.  How you specify
each piece is different in each one, however.

The way that the variables are stored and the parameters and return values
are transferred by the computer varies from language to language as well.  
This variance is  known as
a language's \emph{calling convention}\index{calling
conventions}, because it describes how functions expect
to get and receive data when they are called.\footnote{A 
\emph{convention} is a way of doing things that is standardized,
but not forcibly so.  For example, it is a convention for people to shake
hands when they meet.  If I refuse to shake hands with you, you may think
I don't like you.  Following conventions is important because it makes it
easier for others to understand what you are doing, and makes it easier
for programs written by multiple independent authors to work together.
}

Assembly language can use any calling convention it wants to.  
You can even make one up yourself.  However, if
you want to interoperate with functions written in other 
languages, you have to obey their calling conventions.  We 
will use the calling convention of the C programming language\index{C programming language} 
for our examples because it is the most widely used, and because it is the standard for Linux platforms.

\section{Assembly-Language Functions using the C Calling Convention}
\label{callingwritingassemblyfunctions}

You cannot write assembly-language functions without understanding
how the computer's \emph{stack\index{stack}} works.  Each computer
program that runs uses a region of memory called the stack to enable
functions to work properly.  Think of a stack as a pile of papers
on your desk which can be added to indefinitely.  You generally keep
the things that you are working on toward the top, and you take things
off as you are finished working with them.

Your computer has a stack, too.  The computer's stack\index{stack} lives at the very 
top addresses of memory.  You can push values onto the 
top of the stack through an instruction called
\icode{pushl\index{pushl}}, which pushes either a register or
memory value onto the top of the stack.  Well, we say it's the top, but the
"top" of the stack is actually the bottom of the stack's memory.  
Although this is confusing, the reason for it is that when we think
of a stack of anything - dishes, papers, etc. - we think of adding and 
removing to the top of it.  However,
in memory the stack starts at the top of memory and grows downward due
to architectural considerations.  Therefore, when we refer to the
"top of the stack" remember it's at the bottom of the stack's memory\index{stack memory}.  
You can also pop values off the top using an instruction called
\icode{popl\index{popl}}.
This removes the top value from the stack and places it into a register or memory location of your choosing..

When we push a value onto the stack, the top of the stack moves 
to accomodate the additional value.  We can actually continually 
push values onto the stack and it will keep growing further and
further down in memory until we hit our code or data.
So how do we know where the current "top" of the stack is?  The
stack register\index{stack register}, {\espRegIdx}, always contains a pointer\index{pointer} to the current top of the stack, wherever it is.

Every time we push something onto the stack with \icode{pushl}, 
{\espReg} gets subtracted by 4 so that it points to the new top of the stack 
(remember, each word is four bytes long, and the stack grows downward).  
If we want to remove something from the stack, we simply use the 
\icode{popl} instruction, which adds 4 to {\espReg} and puts the 
previous top value in whatever register you specified.  
\icode{pushl} and \icode{popl} each
take one operand - the register to push onto the stack for 
\icode{pushl}, or receive the data that is popped off the stack
for \icode{popl}.

If we simply want to access the value on the top of the stack without removing it,
we can simply use the {\espRegIdx} register in indirect addressing mode\index{indirect addressing mode}.  For example, the 
following code moves whatever is at the top of the stack into
{\eaxReg}:

\begin{simpletyping}
\begin{lstlisting}
movl ({\espBare}), {\eaxBare}
\end{lstlisting}
\end{simpletyping}

If we were to just do this:

\begin{simpletyping}
\begin{lstlisting}
movl {\espBare}, {\eaxBare}
\end{lstlisting}
\end{simpletyping}

then {\eaxReg} would just hold the pointer to the top of the stack rather than
the value at the top.  Putting {\espReg} in parenthesis causes the computer to
go to indirect addressing mode\index{indirect addressing mode},
and therefore we get the value pointed to by {\espRegIdx}.  If we want to 
access the value right below the top of the stack, we can simply issue this instruction:

\begin{simpletyping}
\begin{lstlisting}
movl 4({\espBare}), {\eaxBare}
\end{lstlisting}
\end{simpletyping}

This instruction uses the base pointer addressing mode\index{base pointer addressing mode}
(see \autoref{dataaccessingmethods})
which simply adds 4 to {\espRegIdx} before looking up the value being pointed to.  

In the C language calling convention\index{C language calling convention}, the stack is the key 
element for implementing a function's local variables, 
parameters, and return address. 

Before executing a function\index{functions},
a program pushes all of the parameters\index{parameters} for the function onto
the stack in the reverse order that they are documented.  Then
the program issues a \icode{call\index{call}} instruction
indicating which function it wishes to start.  The 
\icode{call} instruction does two things.  First
it pushes the address of the next instruction, which is the
return address\index{return address}, onto the stack\index{stack}.  Then it modifies the 
instruction pointer\index{instruction pointer} ({\eipRegIdx})
to point to the start of the function.  So, at the time the
function starts, the stack looks like this (the "top" of the stack is
at the bottom on this example):

%  FIXME - Dominique says "This part can be confusing until one gets to the sample illustration code and then it makes sense." 

\begin{simpletyping}
\begin{lstlisting}
Parameter \#N
...
Parameter 2
Parameter 1
Return Address <--- ({\espBare})
\end{lstlisting}
\end{simpletyping}

Each of the parameters of the function have been pushed onto the stack,
and finally the return address is there.
Now the function itself has some work to do.  

The first thing
it does is save the current base pointer register\index{base pointer register},
 {\ebpRegIdx}, by doing
\icode{pushl {\ebpBare}}.  The base pointer is a special register\index{special register} used for accessing function parameters\index{function parameters} and local variables\index{local variables}.
Next, it copies the stack pointer\index{stack pointer}
to {\ebpRegIdx} by doing \icode{movl {\espBare}, {\ebpBare}}.  This 
allows you to be able to access the function parameters
as fixed indexes from the base pointer.  You may think that you can
use the stack pointer for this.  However, during your
program you may do other things with the stack such as pushing
arguments to other functions.

Copying the stack pointer into
the base pointer at the beginning of a function allows you to always 
know where your parameters are (and as we will see, local variables too),
even while you may be pushing things on and off the stack.  {\ebpRegIdx}
will always be where the stack pointer was at the beginning of the function,
so it is more or less a constant reference to the \emph{stack frame\index{stack frame}} (the stack frame
consists of all of the stack variables used within a function, including 
parameters\index{parameters}, local variables\index{local variables}, and the return address\index{return address}).

At this point, the stack looks like this:

\begin{simpletyping}
\begin{lstlisting}
Parameter \#N   <--- N*4+4({\ebpBare})
...
Parameter 2    <--- 12({\ebpBare})
Parameter 1    <--- 8({\ebpBare})
Return Address <--- 4({\ebpBare})
Old {\ebpBare}       <--- ({\espBare}) and ({\ebpBare})
\end{lstlisting}
\end{simpletyping}

As you can see, each parameter can be accessed using base pointer addressing mode\index{base pointer addressing mode} using the {\ebpRegIdx} register.

Next, the function reserves space on the stack for any local
variables\index{local variables} it needs.
This is done by simply moving the stack pointer\index{stack pointer} out of the way.
Let's say that we are going to need two words of memory
to run a function.  We can simply move the stack pointer down two
words to reserve the space.  This is done like this:

\begin{simpletyping}
\begin{lstlisting}
subl \$8, {\espBare}
\end{lstlisting}
\end{simpletyping}

This subtracts 8 from {\espReg} (remember, a word is four bytes 
long).\footnote{Just a reminder - the dollar sign in
front of the eight indicates immediate mode addressing\index{immediate mode addressing}, meaning
that we subtract the number 8 itself from {\espReg} rather than the value at
address 8.}  This way, we can use the stack for
variable storage without worring about clobbering them with pushes
that we may make for function calls.  Also, since it is allocated on 
the stack frame for this function call, the variable will only be 
alive during this function.  When we return, the stack frame will go
away, and so will these variables.  That's why they are called local -
they only exist while this function is being called.

Now we have two words for local storage.  Our stack now looks like this:

\begin{simpletyping}
\begin{lstlisting}
Parameter \#N     <--- N*4+4({\ebpBare})
...
Parameter 2      <--- 12({\ebpBare})
Parameter 1      <--- 8({\ebpBare})
Return Address   <--- 4({\ebpBare})
Old {\ebpBare}         <--- ({\ebpBare})
Local Variable 1 <--- -4({\ebpBare})
Local Variable 2 <--- -8({\ebpBare}) and ({\espBare})
\end{lstlisting}
\end{simpletyping}

So we can now access all of the data we need for this function
by using base pointer addressing\index{base pointer addressing mode} using different offsets from {\ebpRegIdx}.
{\ebpRegIdx} was made specifically for this purpose, 
which is why it is called the base pointer\index{base pointer register}.  You can use other
registers in base pointer addressing mode, but the x86 architecture
makes using the {\ebpRegIdx} register a lot faster.

\index{static variables}
\index{global variables} 
Global variables and static variables are accessed just like the memory
we have been accessing memory in previous chapters.  The only difference
between the global and static variables is that static variables are 
only used by one function, while global variables are used by 
many functions. Assembly language treats them exactly the same, 
although most other languages distinguish them.

When a function is done executing, it does three things:

\begin{enumerate}\item It stores its return value in {\eaxRegIdx}. 
\item It resets the stack to what it was when it was called (it gets rid of the current stack frame\index{stack frame} and puts the stack frame of the calling code back into effect). 
\item It returns control back to wherever it was called from.  This is done using the 
\icode{ret\index{ret}} instruction, which pops whatever
value is at the top of the stack, and sets the instruction pointer\index{instruction pointer}, 
{\eipRegIdx}, to that value. 
\end{enumerate}

So, before a function returns control to the code that called it, it
must restore the previous stack frame.  Note also that without doing this,
\icode{ret} wouldn't work, because in our current stack
frame, the return address is not at the top of the stack.  Therefore, before
we return, we have to reset the stack pointer\index{stack pointer} {\espRegIdx} and base pointer\index{base pointer} {\ebpRegIdx} to what they
were when the function began.

Therefore to return from the function you have to do the following:

\begin{simpletyping}
\begin{lstlisting}
movl {\ebpBare}, {\espBare}
popl {\ebpBare}
ret
\end{lstlisting}
\end{simpletyping}

\emph{At this point, you should consider all local variables to be 
disposed of.}  The reason is that after you move the stack
pointer back, future stack pushes will likely overwrite everything
you put there.  Therefore, you should never save the address
of a local variable\index{local variables} past the life of the function it was 
created in, or else it will be overwritten after the life of its stack 
frame ends.  

Control has now been handed back to the calling code,
which can now examine {\eaxRegIdx} for the return value\index{return value}.  The calling
code also needs to pop off all of the parameters it 
pushed onto the stack in order to get the stack pointer\index{stack pointer}
back where it was (you can also simply add 4 * number of parameters
to {\espRegIdx} using the \icode{addl} instruction, if 
you don't need the values of the parameters anymore).\footnote{This is not always strictly needed unless you are saving registers on the stack before a function call.  The base pointer keeps the stack frame in a reasonably consistent state.  However, it is still a good idea, and is absolutely necessary if you are temporarily saving registers on the stack..}

\begin{sidebar}[Destruction of Registers]
When you call a function\index{functions}, you should assume that everything
currently in your registers\index{registers} will be wiped out.  The only
register that is guaranteed to be left with the value it
started with are {\ebpRegIdx} and a few others (the Linux C calling 
convention requires functions to preserve the values of {\ebxRegIdx}, 
{\ediRegIdx}, and {\esiRegIdx} if they are altered - this is not
strictly held during this book because these programs are self-contained
and not called by outside functions).  {\ebxReg} also has some other uses
in position-independent code, which is not covered in this book.
{\eaxRegIdx} is guaranteed to be overwritten with the return value,
and the others likely are.  If there are registers you want
to save before calling a function, you need to save them by
pushing them on the stack\index{stack} before pushing the function's 
parameters.  You can then pop them back off in reverse order
after popping off the parameters.  Even if you know a function
does not overwrite a register you should save it, because
future versions of that function may.

Note that in Linux assembly language,
functions are 

Other languages' calling 
conventions\index{calling conventions}
may be different.  For example, other calling conventions may
place the burden on the function to save any registers it uses.  Be sure
to check to make sure the calling conventions of your languages 
are compatible before trying to mix languages.  Or in the case of assembly
language, be sure you know how to call the other language's functions.
\end{sidebar}

\begin{sidebar}[Extended Specification]
Details of the C language calling convention\index{calling convention} 
(also known as the ABI\index{ABI}, or 
Application Binary Interface\index{Application Binary Interface}) is available online.  We have oversimplified and left
out several important pieces to make this simpler for new programmers.
For full details, you should check out the documents available at
http://www.linuxbase.org/spec/refspecs/  Specifically, you should look
for the \documentname{System V Application Binary Interface - Intel386
Architecture Processor Supplement}.
\end{sidebar}

\section{A Function Example}

Let's take a look at how a function call\index{function call} works in a real program.  The
function we are going to write is the \icode{power}
function.  We will give the power function two parameters -
the number and the power we want to raise it to.  For example,
if we gave it the parameters 2 and 3, it would raise 2 to the
power of 3, or 2*2*2, giving 8.  In order to make this
program simple, we will only allow numbers 1 and greater.

The following is the code for the complete program.  As usual,
an explanation follows.  Name the file \icode{power.s}.

\begin{simpletyping}
\lstinputlisting{power.s}
\end{simpletyping}

Type in the program, assemble it, and run it.  Try calling
power for different values, but remember that the result
has to be less than 256 when it is passed back to the operating
system.  Also try subtracting the results of the two 
computations.  Try adding a third call to the 
\icode{power} function, and add its result
back in.  

The main program code is pretty simple.  You push the 
arguments onto the stack, call the function, and then move
the stack pointer back.  The result is stored in {\eaxReg}.
Note that between the two calls to \icode{power},
we save the first value onto the stack.  This is because the
only register that is guaranteed to be saved is {\ebpRegIdx}.
Therefore we push the value onto the stack, and pop the value 
back off after the second function call is complete.

Let's look at how the function itself is written.  Notice
that before the function, there is documentation as to
what the function does, what its arguments are, and
what it gives as a return value.  This is useful for 
programmers who use this function.  This is the function's
interface.  This lets the programmer know what values are
needed on the stack, and what will be in {\eaxReg} at the end.

We then have the following line:

\begin{simpletyping}
\begin{lstlisting}
	.type power,@function
\end{lstlisting}
\end{simpletyping}

\index{.type}
\index{@functions}
This tells the linker that the symbol \icode{power}
should be treated as a function.  Since this program
is only in one file, it would work just the same with this
left out.  However, it is good practice.

After that, we define the value of the \icode{power} label:

\begin{simpletyping}
\begin{lstlisting}
power:
\end{lstlisting}
\end{simpletyping}

As mentioned previously, this defines the symbol 
\icode{power} to be the address where the instructions
following the label begin.  This is how 
\icode{call power} works.  It transfers control to
this spot of the program.  The difference between 
\icode{call\index{call}} and \icode{jmp\index{jmp}} is that 
\icode{call} also pushes the return address onto
the stack so that the function can return, while the 
\icode{jmp} does not.

Next, we have our instructions to set up our function:

\begin{simpletyping}
\begin{lstlisting}
	pushl {\ebpBare}
	movl  {\espBare}, {\ebpBare}
	subl  \$4, {\espBare}
\end{lstlisting}
\end{simpletyping}

At this point, our stack looks like this:

\begin{simpletyping}
\begin{lstlisting}
Base Number    <--- 12({\ebpBare})
Power          <--- 8({\ebpBare})
Return Address <--- 4({\ebpBare})
Old {\ebpBare}       <--- ({\ebpBare})
Current result <--- -4({\ebpBare}) and ({\espBare})
\end{lstlisting}
\end{simpletyping}

Although we could use a register for temporary storage, this
program uses a 
local variable\index{local variables}
in order to show how to set it
up.  Often times there just aren't enough registers to store
everything, so you have to offload them into local variables.
Other times, your function will need to call another function
and send it a pointer to some of your data.  You can't have
a pointer\index{pointer} 
to a register\index{register}, 
so you have to store it in a 
local variable in order to send a pointer to it.

Basically, what the program does is start with the base number,
and store it both as the multiplier (stored in {\ebxReg}) and the 
current value (stored in -4({\ebpBare})).  It also has the power
stored in {\ecxReg}  It then continually 
multiplies the current value by the multiplier, decreases 
the power, and leaves the loop if the power (in {\ecxReg}) gets down to 1.

By now, you should be able to go through the program without
help.  The only things you should need to know is that
\icode{imull\index{imull}} does integer multiplication and stores
the result in the second operand, and \icode{decl\index{decl}}
decreases the given register by 1.  For more information on these
and other instructions, see \autoref{instructionsappendix}

A good project to try now is to extend the program so it
will return the value of a number if the power is 0 (hint,
anything raised to the zero power is 1).  Keep trying.
If it doesn't work at first, try going through your program
by hand with a scrap of paper, keeping track of where
{\ebpReg} and {\espReg} are pointing, what is on the stack, and what the
values are in each register.

\section{Recursive Functions}
\label{recursivefunctions}

The next program will stretch your brains even
more.  The program will compute the 
\emph{factorial} of a number.  A
factorial is the product of a number and all the numbers between it
and one.  For example, the factorial of 7 is 7*6*5*4*3*2*1, and the
factorial of 4 is 4*3*2*1.  Now, one thing you might notice is that
the factorial of a number is the same as the product of a number and
the factorial just below it.  For example, the factorial of 4 is
4 times the factorial of 3.  The factorial of 3 is 3 times the factorial
of 2.  2 is 2 times the factorial of 1.  The factorial of 1 is 1.  
This type of definition is called a recursive\index{recursive} definition.  That means,
the definition of the factorial function\index{functions} includes the factorial function itself. 
However, since all functions need to end, a recursive definition must
include a \emph{base case\index{base case}}.  The base case is the
point where recursion will stop.  Without a base case, the function would
go on forever calling itself until it eventually ran out of stack space.  
In the case of the factorial, the base case
is the number 1.  When we hit the number 1, we don't run the factorial
again, we just say that the factorial of 1 is 1.  So, let's run through
what we want the code to look like for our factorial 
function:

\begin{enumerate}\item Examine the number 
\item Is the number 1? 
\item If so, the answer is one 
\item Otherwise, the answer is the number times the factorial of the number minus one 
\end{enumerate}

This would be problematic if we didn't have local variables\index{local variables}.
In other programs, storing values in global variables worked fine.  However,
global variables only provide one copy of each variable.  In this program,
we will have multiple copies of the function running at the same time, all
of them needing their own copies of the data!\footnote{By "running
at the same time" I am talking about the fact that one will not have
finished before a new one is activated.  I am not implying that their
instructions are running at the same time.}
Since local variables exist on the stack frame, and each function call
gets its own stack frame\index{stack frame}, we are okay.

Let's look at the code to see how this works:

\begin{simpletyping}
\lstinputlisting{factorial.s}
\end{simpletyping}

Assemble, link, and run it with these commands:

\begin{simpletyping}
\begin{lstlisting}
as factorial.s -o factorial.o
ld factorial.o -o factorial
./factorial
echo \$?
\end{lstlisting}
\end{simpletyping}

This should give you the value 24.  24 is the factorial of 4, you can
test it out yourself with a calculator: 4 * 3 * 2 * 1 = 24.

I'm guessing you didn't understand the whole code listing.  Let's go
through it a line at a time to see what is happening. 

\begin{simpletyping}
\begin{lstlisting}
\_start:
	pushl \$4
	call factorial
\end{lstlisting}
\end{simpletyping}

Okay, this program is intended to compute the factorial of the number 
4.  When programming functions, you are supposed to put the
parameters\index{parameters} of the function on the top of the stack right before
you call it.  Remember, a function's \emph{parameters\index{parameters}}
are the data that you want the function to work with.  In this case,
the factorial function takes 1 parameter - the number you want the
factorial of.  

The \icode{pushl\index{pushl}}
instruction puts the given value at the top of the stack.  
The \icode{call\index{call}} instruction then makes the function call.

Next we have these lines:

\begin{simpletyping}
\begin{lstlisting}
        addl  \$4, {\espBare}
        movl  {\eaxBare}, {\ebxBare}
        movl  \$1, {\eaxBare}
        int   \$0x80
\end{lstlisting}
\end{simpletyping}

This takes place after \icode{factorial} has finished and computed
the factorial of 4 for us.  Now we have to clean up the stack.
The \icode{addl} instruction moves the stack pointer back
to where it was before we pushed the \icode{\$4} onto the stack.
You should always clean up your stack parameters after a function call returns.

The next instruction moves {\eaxReg} to
{\ebxReg}.  What's in {\eaxRegIdx}?  It is
\icode{factorial}'s return value\index{return value}.  
In our case, it is the value of the factorial function.  With 4 as our
parameter, 24 should be our return value.  Remember, return values are
always stored in {\eaxRegIdx}.  We want to return this value as the
status code to the operating system.  However,
Linux requires that the program's exit status be stored in
{\ebxRegIdx}, not {\eaxReg}, so we have to
move it.  Then we do the standard exit system call.

The nice thing about function calls is that:

\begin{itemize}
\item Other programmers don't have to know anything about them except its arguments to use them. 
\item They provide standardized building blocks from which you can form a program. 
\item They can be called multiple times and from multiple locations and they always know how to get back to where they were since \icode{call\index{call}} pushes the return address onto the stack. 
\end{itemize}

These are the main advantages of functions.  
Larger programs also use functions to break 
down complex pieces of code into smaller, simpler ones.  In fact, almost
all of programming is writing and calling functions. 

Let's now take a look at how the \icode{factorial} 
function itself is implemented.

Before the function starts, we have this directive:

\begin{simpletyping}
\begin{lstlisting}
	.type factorial,@function
factorial:
\end{lstlisting}
\end{simpletyping}

The \icode{.type\index{.type}} directive tells the linker that 
\icode{factorial} is a function.  This isn't really needed
unless we were using \icode{factorial} in other programs.
We have included it for completeness.  The line that says 
\icode{factorial:} gives the symbol \icode{factorial}
the storage location of the next instruction.  That's how 
\icode{call} knew where to go when we said 
\icode{call factorial}.  

The first real instructions of the function are:

\begin{simpletyping}
\begin{lstlisting}
	pushl {\ebpBare}
	movl  {\espBare}, {\ebpBare}
\end{lstlisting}
\end{simpletyping}

As shown in the previous program, this creates the stack frame\index{stack frame}
for this function.  These two lines will be the way you should
start every function.

The next instruction is this:

\begin{simpletyping}
\begin{lstlisting}
	movl  8({\ebpBare}), {\eaxBare}
\end{lstlisting}
\end{simpletyping}

This uses base pointer addressing\index{base pointer addressing mode} to move the first parameter\index{parameter}
of the function into {\eaxReg}.  Remember, \icode{({\ebpBare})}
has the old {\ebpReg}, \icode{4({\ebpBare})} has the return address,
and \icode{8({\ebpBare})} is the location of the first parameter 
to the function.  If you think back, this will be the value 4
on the first call, since
that was what we pushed on the stack before calling the function the
first time (with \icode{pushl \$4}).  As this function calls itself,
it will have other values, too.

Next, we check to see if we've hit our base case (a parameter of 1).  If
so, we jump to the instruction at the label \icode{end\_factorial},
where it will be returned.  It's already in {\eaxReg} which
we mentioned earlier is where you put return values\index{return values}.  That is accomplished
by these lines:

\begin{simpletyping}
\begin{lstlisting}
	cmpl \$1, {\eaxBare}
	je end\_factorial
\end{lstlisting}
\end{simpletyping}

If it's not our base case, what did we say we would do?  We would call
the \icode{factorial} function again with our parameter minus
one.  So, first we decrease {\eaxReg} by one:

\begin{simpletyping}
\begin{lstlisting}
	decl {\eaxBare}
\end{lstlisting}
\end{simpletyping}

\icode{decl\index{decl}} stands for decrement.  It subtracts 1 from
the given register or memory location ({\eaxReg} in our case).  
\icode{incl\index{incl}} is the
inverse - it adds 1.  After decrementing {\eaxReg}
we push it onto the stack since it's going to be the parameter of
the next function call.  And then we call \icode{factorial} again!

\begin{simpletyping}
\begin{lstlisting}
	pushl {\eaxBare}
	call factorial
\end{lstlisting}
\end{simpletyping}

Okay, now we've called \icode{factorial}.  One thing to remember 
is that after a function call, we can never know what the registers are
(except \icode{{\espBare}} and \icode{{\ebpBare}}).  So
even though we had the value we were called with in \icode{{\eaxBare}},
it's not there any more.  Therefore, we need pull it off the stack
from the same place we got it the first time (at 
\icode{8({\ebpBare})}).
So, we do this:

\begin{simpletyping}
\begin{lstlisting}
	movl 8({\ebpBare}), {\ebxBare}
\end{lstlisting}
\end{simpletyping}

Now, we want to multiply that number with the result of the
factorial function.  If you remember our previous discussion,
the result of functions are left in {\eaxReg}.
So, we need to multiply {\ebxReg} with {\eaxReg}.
This is done with this instruction:

\begin{simpletyping}
\begin{lstlisting}
	imull {\ebxBare}, {\eaxBare}
\end{lstlisting}
\end{simpletyping}

This also stores the result in {\eaxReg}, which is
exactly where we want the return value for the function to be!  Since
the return value\index{return value} is in place
we just need to leave the function.  If you remember, at the
start of the function we pushed {\ebpReg}, and
moved {\espReg} into {\ebpReg} to create the current stack frame.  Now
we reverse the operation to destroy the current stack frame and 
reactivate the last one:

\begin{simpletyping}
\begin{lstlisting}
end\_factorial:
	movl {\ebpBare}, {\espBare}
	popl {\ebpBare}
\end{lstlisting}
\end{simpletyping}

Now we're already to return, so we issue the following command

\begin{simpletyping}
\begin{lstlisting}
	ret
\end{lstlisting}
\end{simpletyping}

This pops the top value off of the stack, and then jumps to it.  If
you remember our discussion about \icode{call}, we said
that \icode{call\index{call}} first pushed the address of the
next instruction onto the stack before it jumped to the beginning
of the function.  So, here we pop it back off so we can return there.
The function is done, and we have our answer!  

Like our previous program, you should look over the
program again, and make sure you know what everything does.
Look back through this section and the previous sections for the 
explanation of anything you don't understand.  Then, take a 
piece of paper, and go through the program
step-by-step, keeping track of what the values of the registers
are at each step, and what values are on the stack.  Doing this
should deepen your understanding of what is going on.

\section{Review}

\section{Know the Concepts}

\begin{itemize}\item What are primitives? 
\item What are calling conventions? 
\item What is the stack? 
\item How do \icode{pushl} and \icode{popl} affect the stack?  What special-purpose register do they affect? 
\item What are local variables and what are they used for? 
\item Why are local variables so necessary in recursive functions? 
\item What are {\ebpReg} and {\espReg} used for? 
\item What is a stack frame? 
\end{itemize}

\section{Use the Concepts}
\label{functionsreviewuseconcepts}

\begin{itemize}\item Write a function called \icode{square} which receives one argument and returns the square of that argument. 
\item Write a program to test your \icode{square} function. 
\item Convert the maximum program given in \autoref{maximum} so that it is a function which takes a pointer to several values and returns their maximum.  Write a program that calls maximum with 3 different lists, and returns the result of the last one as the program's exit status code. 
\item Explain the problems that would arise without a standard calling convention. 
\end{itemize}

\section{Going Further}

\begin{itemize}\item Do you think it's better for a system to have a large set of primitives or a small one, assuming that the larger set can be written in terms of the smaller one? 
\item The factorial function can be written non-recursively.  Do so. 
\item Find an application on the computer you use regularly.  Try to locate a specific feature, and practice breaking that feature out into functions.  Define the function interfaces between that feature and the rest of the program. 
\item Come up with your own calling convention.  Rewrite the programs in this chapter using it.  An example of a different calling convention would be to pass parameters in registers rather than the stack, to pass them in a different order, to return values in other registers or memory locations.  Whatever you pick, be consistent and apply it throughout the whole program. 
\item Can you build a calling convention without using the stack?  What limitations might it have? 
\item What test cases should we use in our example program to check to see if it is working properly? 
\end{itemize}

