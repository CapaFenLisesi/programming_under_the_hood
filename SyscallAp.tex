\chapter{Important System Calls}
\label{syscallap}

% 
% 
% Copyright 2002 Jonathan Bartlett
% 
% Permission is granted to copy, distribute and/or modify this
% document under the terms of the GNU Free Documentation License,
% Version 1.1 or any later version published by the Free Software
% Foundation; with no Invariant Sections, with no Front-Cover Texts,
% and with no Back-Cover Texts.  A copy of the license is included in fdl.xml
% 

These are some of the more important system calls to use when dealing with
Linux.  For most cases, however, it is best to use library functions rather
than direct system calls, because the system calls were designed to be
minimalistic while the library functions were designed to be easy to
program with.  For information about the Linux C library, see the manual
at http://www.gnu.org/software/libc/manual/

Remember that {\eaxRegIdx} holds the system call numbers, and that the
return values and error codes are also stored in {\eaxReg}.
\index{\%eax}
\index{\%ebx}
\index{\%ecx}
\index{\%edx}

\begin{table}[h]
\begin{tabular}{FIXME}

{\eaxReg} & Name & {\ebxReg} & {\ecxReg} & {\edxReg} & Notes & 
1 & exit\index{exit} & return value (int) &  &  & Exits the program & 
3 & read\index{read} & file descriptor & buffer start & buffer size (int) & Reads into the given buffer & 
4 & write\index{write} & file descriptor & buffer start & buffer size (int) & Writes the buffer to the file descriptor & 
5 & open\index{open} & null-terminated file name & option list & permission mode & Opens the given file.  Returns the file descriptor or an error number. & 
6 & close\index{close} & file descriptor &  &  & Closes the give file descriptor & 
12 & chdir\index{chdir} & null-terminated directory name &  &  & Changes the current directory of your program. & 
19 & lseek\index{lseek} & file descriptor & offset & mode & Repositions where you are in the given file.  The mode (called the "whence") should be 0 for absolute positioning, and 1 for relative positioning. & 
20 & getpid\index{getpid} &  &  &  & Returns the process ID of the current process. & 
39 & mkdir\index{mkdir} & null-terminated directory name & permission mode &  & Creates the given directory.  Assumes all directories leading up to it already exist. & 
40 & rmdir\index{rmdir} & null-terminated directory name &  &  & Removes the given directory. & 
41 & dup\index{dup} & file descriptor &  &  & Returns a new file descriptor that works just like the existing file descriptor. & 
42 & pipe\index{pipe} & pipe array &  &  & Creates two file descriptors, where writing on one produces data to read on the other and vice-versa.  {\ebxRegIdx} is a pointer to two words of storage to hold the file descriptors. & 
45 & brk\index{brk} & new system break &  &  & Sets the system break (i.e. - the end of the data section).  If the system break is 0, it simply returns the current system break. & 
54 & ioctl\index{ioctl} & file descriptor & request & arguments & This is used to set parameters on device files.  Its actual usage varies based on the type of file or device your descriptor references. & 
% 
% % & % & % & % & % & % & % 
\end{tabular}
\caption{Important Linux System Calls}
\end{table}

A more complete listing of system calls, along with additional information
is available at http://www.lxhp.in-berlin.de/lhpsyscal.html
You can also get more information about a system call by typing in 
\icode{man 2 SYSCALLNAME} which will return you the information 
about the system call from section 2 of the UNIX manual\index{UNIX manual}.  However, this 
refers to the usage of the system call from the C programming language\index{C programming language}, 
and may or may not be directly helpful.

For information on how system calls\index{system calls} are implemented on Linux, see the Linux Kernel 2.4 Internals section on how system calls are implemented 
at http://www.faqs.org/docs/kernel\_2\_4/lki-2.html\#ss2.11

