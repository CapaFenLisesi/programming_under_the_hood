\chapter{GUI Programming}

% 
% 
% Copyright 2002 Jonathan Bartlett
% 
% Permission is granted to copy, distribute and/or modify this
% document under the terms of the GNU Free Documentation License,
% Version 1.1 or any later version published by the Free Software
% Foundation; with no Invariant Sections, with no Front-Cover Texts,
% and with no Back-Cover Texts.  A copy of the license is included in fdl.xml
% 

\section{Introduction to GUI Programming}

The purpose of this appendix is not to teach you how to
do Graphical User Interfaces.  It is simply meant to show
how writing graphical applications is the same as writing
other applications, just using an additional library to
handle the graphical parts.  As a programmer you need
to get used to learning new libraries.  Most of your time
will be spent passing data from one library to another.

\section{The GNOME Libraries}

The GNOME\index{GNOME} projects is one of several projects to provide
a complete desktop to Linux users.  The GNOME project includes
a panel to hold application launchers and mini-applications
called applets, several standard applications to do things
such as file management, session management, and configuration,
and an API for creating applications which fit in with the
way the rest of the system works.

One thing to notice about the GNOME libraries is that they
constantly create and give you pointers to large data 
structures, but you never need to know how they are laid
out in memory.  All manipulation of the GUI data structures
are done entirely through function calls.  This is a 
characteristic of good library design.  Libraries change
from version to version, and so does the data that each
data structure holds.  If you had to access and manipulate
that data yourself, then when the library is updated you
would have to modify your programs to work with the new library,
or at least recompile them.  When you access the data through
functions, the functions take care of knowing where in the
structure each piece of data is.  The pointers you receive
from the library are \emph{opaque} - you don't
need to know specifically what the structure they are pointing
to looks like, you only need to know the functions that will
properly manipulate it.  When designing libraries, even for
use within only one program, this is a good practice to keep
in mind.

This chapter will not go into details about how GNOME works.
If you would like to know more, visit the GNOME developer web
site at http://developer.gnome.org/.
This site contains tutorials, mailing lists, API documentation,
and everything else you need to start programming in the GNOME
environment.  

\section{A Simple GNOME Program in Several Languages}

This program will simply show a Window that has a button to
quit the application.  When that button is clicked it will
ask you if you are sure, and if you click yes it will close
the application.  To run this program, type in the following
as \icodefilename{gnome-example.s}:

\begin{simpletyping}
\lstinputlisting{gnome-example.s}
\end{simpletyping}

To build this application, execute the following commands:

\begin{simpletyping}
\begin{lstlisting}
as gnome-example.s -o gnome-example.o
gcc gnome-example.o `gnome-config --libs gnomeui` \\
    -o gnome-example
\end{lstlisting}
\end{simpletyping}

Then type in \icode{./gnome-example} to run it.

This program, like most GUI\index{GUI} programs, makes heavy use of 
passing pointers to functions as parameters.  In this program
you create widgets with the GNOME functions and then you
set up functions to be called when certain events happen.
These functions are called \emph{callback}
functions.  All of the event processing is handled by 
the function \icode{gtk\_main}, so you don't
have to worry about how the events are being processed.
All you have to do is have callbacks set up to wait for
them.

Here is a short description of all of the GNOME functions
that were used in this program:

\begin{description}
\item[gnome\_init] Takes the command-line arguments, argument count, application
id, and application version and initializes the GNOME 
libraries.
\item[gnome\_app\_new] Creates a new application window, and returns a pointer to it.
Takes the application id and the window title as arguments.
\item[gtk\_button\_new\_with\_label] Creates a new button and returns a pointer to it.
Takes one argument - the text that is in the button.
\item[gnome\_app\_set\_contents] This takes a pointer to the gnome application window
and whatever widget you want (a button in this case) and 
makes the widget be the contents of the application window
\item[gtk\_widget\_show] This must be called on every widget created (application 
window, buttons, text entry boxes, etc) in order
for them to be visible.  However, in order for a given 
widget to be visible, all of its parents must be visible 
as well.
\item[gtk\_signal\_connect] This is the function that connects widgets and their
signal handling callback functions.  This function takes
the widget pointer, the name of the signal, the callback
function, and an extra data pointer.  After this function
is called, any time the given event is triggered, the 
callback will be called with the widget that produced
the signal and the extra data pointer.  In this application,
we don't use the extra data pointer, so we just set it to
NULL, which is 0.
\item[gtk\_main] This function causes GNOME to enter into its main loop.
To make application programming easier, GNOME handles the
main loop of the program for us.  GNOME will check for
events and call the appropriate callback functions when
they occur.  This function will continue to process events
until \icode{gtk\_main\_quit} is called by a signal
handler.
\item[gtk\_main\_quit] This function causes GNOME to exit its main loop at the
earliest opportunity.
\item[gnome\_message\_box\_new] This function creates a dialog window containing a question
and response buttons.  It takes as parameters the message
to display, the type of message it is (warning, question, etc),
and a list of buttons to display.  The final parameter should
be NULL to indicate that there are no more buttons to display.
\item[gtk\_window\_set\_modal] This function makes the given window a modal window.  In
GUI programming, a modal window is one that prevents event
processing in other windows until that window is closed.
This is often used with Dialog windows.
\item[gnome\_dialog\_run\_and\_close] This function takes a dialog pointer (the pointer returned
by \icode{gnome\_message\_box\_new} can be used
here) and will set up all of the appropriate signal handlers
so that it will run until a button is pressed.  At that
time it will close the dialog and return to you which button
was pressed.  The button number refers to the order in which
the buttons were set up in \icode{gnome\_message\_box\_new}.
\end{description}

The following is the same program written in the C language.
Type it in as \icodefilename{gnome-example-c.c}:

\begin{simpletyping}
\lstinputlisting{gnome-example-c-c}
\end{simpletyping}

To compile it, type

\begin{simpletyping}
\begin{lstlisting}
gcc gnome-example-c.c `gnome-config --cflags \\
    --libs gnomeui` -o gnome-example-c
\end{lstlisting}
\end{simpletyping}

Run it by typing \icode{./gnome-example-c}.

Finally, we have a version in Python.  Type it in as gnome-example.py:

\begin{simpletyping}
\lstinputlisting{gnome-example-py}
\end{simpletyping}

To run it type \icode{python gnome-example.py}.

\section{GUI Builders}

In the previous example, you have created the user-interface
for the application by calling the create functions for each
widget and placing it where you wanted it.  However, this can
be quite burdensome for more complex applications.  Many 
programming environments, including GNOME, have programs
called GUI builders that can be used to automatically create
your GUI for you.  You just have to write the code for the
signal handlers and for initializing your program.  The
main GUI builder\index{GUI builder} for GNOME applications is called GLADE.
GLADE ships with most Linux distributions.

There are GUI builders for most programming environments.
Borland has a range of tools that will build GUIs quickly 
and easily on Linux and Win32\index{Win32} systems.  The KDE environment 
has a tool called QT Designer\index{QT Designer} which helps you automatically 
develop the GUI for their system.  

There is a broad range
of choices for developing graphical applications, but 
hopefully this appendix gave  you a taste of what GUI programming
is like.

