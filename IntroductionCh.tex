\chapter{Introduction}

% 
% 
% Copyright 2002 Jonathan Bartlett
% 
% Permission is granted to copy, distribute and/or modify this
% document under the terms of the GNU Free Documentation License,
% Version 1.1 or any later version published by the Free Software
% Foundation; with no Invariant Sections, with no Front-Cover Texts,
% and with no Back-Cover Texts.  A copy of the license is included in fdl.xml
% 

\section{Welcome to Programming}

I love
\index{programming}programming.  I
enjoy the challenge to not only make a working program, but to do so
with style.  Programming is like poetry.  It conveys a message, not
only to the computer, but to those who modify and use your program.
With a program, you build your own world with your own rules.  You
create your world according to your conception of both the problem and
the solution.  Masterful programmers create worlds with programs that
are clear and succinct, much like a poem or essay.

One of the greatest programmers, Donald Knuth, 
describes programming not as telling a computer how to do something,
but telling a person how they would instruct a computer to do something.
The point is that programs are meant to be read by people, not
just computers.  Your programs will be modified and updated by others
long after you move on to other projects.  Thus, programming is not
as much about communicating to a computer as it is communicating to
those who come after you.  A programmer is a problem-solver, a poet, and an
instructor all at once.  Your goal is to solve the problem at hand,
doing so with balance and taste, and teach your solution to
future programmers.  I hope that this book can teach at least some
of the poetry and magic that makes computing exciting.

 

Most introductory books on programming frustrate me to no end.  At the
end of them you can still ask "how does the computer really work?" and
not have a good answer.  They tend to pass over topics that are 
difficult even though they are important.  I will take you through
the difficult issues because that is the only way to move on to
masterful programming.  My goal is to take you from knowing nothing about
programming to understanding how to think, write, and learn like
a programmer.  You won't know everything, but you will have a background
for how everything fits together.  At the end of this book, you should 
be able to do the following:

\begin{itemize}
\item Understand how a program works and interacts with other programs 
\item Read other people's programs and learn how they work 
\item Learn new programming languages quickly 
\item Learn advanced concepts in computer science quickly 
\end{itemize}

I will not teach you everything.  Computer science is a massive
field, especially when you combine the theory with the practice of computer
programming.  However, I will attempt to get you started on the 
foundations so you can easily go wherever you want afterwards.

There is somewhat of a chicken and egg problem in teaching programming,
especially assembly language.  There is a lot to learn - it is almost too
much to learn almost at all at once.  However, each piece depends on all 
the others, which makes learning it a piece at a time difficult.
Therefore, you must be patient with yourself and the computer while 
learning to program.  If you don't understand something the first time,
reread it.  If you still don't understand it, it is sometimes best to
take it by faith and come back to it later.  Often after more exposure
to programming the ideas will make more sense.  Don't get discouraged.
It's a long climb, but very worthwhile.

At the end of each chapter are three sets of review exercises.  The first
set is more or less regurgitation - they check to see if can you give 
back what you learned in the chapter.  The second set contains application 
questions - they check to see if you can apply what you learned to solve 
problems.  The final set is to see if you are capable of broadening your
horizons.  Some of these questions may not be answerable until later in
the book, but they give you some things to think about.  Other questions
require some research into outside sources to discover the answer.  Still
others require you to simply analyze your options and explain a best solution.
Many of the questions don't have right or wrong answers, but that doesn't mean
they are unimportant.  Learning the issues involved in programming, learning
how to research answers, and learning how to look ahead are all a major
part of a programmer's work. 

If you have problems that you just can't get past, there is a mailing list
for this book where readers can discuss and get help with what they are
reading.  The address is \icode{pgubook-readers@nongnu.org}.
This mailing list is open for any type of question or discussion along the
lines of this book.  You can subscribe to this list by going to http://mail.nongnu.org/mailman/listinfo/pgubook-readers.

If you are thinking of using this book for a class on computer programming but
do not have access to Linux computers for your students, I highly suggest you
try to find help from the K-12 Linux Project.  Their website is at 
http://www.k12linux.org/ and they have a helpful and responsive mailing list
available.

\section{Your Tools}

This book teaches assembly language for x86 processors and the GNU/Linux
operating system.  Therefore we will be giving all of 
the examples using the GNU/Linux\index{GNU/Linux} standard GCC tool set. 
If you are not familiar with GNU/Linux and the GCC tool set, they will
be described shortly.  If you are new to Linux, you 
should check out the guide available at 
http://rute.sourceforge.net/\footnote{This is quite a large 
document.  You certainly don't need to know everything to get started 
with this book.  You simply need to know how to navigate from the command
line and how to use an editor like \icode{pico}, 
\icode{emacs}, or \icode{vi} 
(or others).}
What I intend to show you is more about programming in general than using
a specific tool set on a specific platform, but standardizing on one 
makes the task much easier.

Those new to Linux should also try to get involved in their local GNU/Linux
User's Group.  User's Group members are usually very helpful for new people,
and will help you from everything from installing Linux to learning to
use it most efficiently.  A listing of GNU/Linux User's Groups is available
at http://www.linux.org/groups/

All of these
programs have been tested using \documentname{Red Hat Linux 8.0}, 
and should work with any other GNU/Linux distribution, too.\footnote{By 
"GNU/Linux distribution", I mean an x86 GNU/Linux distribution.  GNU/Linux 
distributions for the Power Macintosh, the Alpha processor, or other 
processors will not work with this book.}  They will
not work with non-Linux operating systems such as BSD or other systems.
However, all of the \emph{skills} learned in this book 
should be easily transferable to any other system.  

If you do not have access to a GNU/Linux\index{GNU/Linux} machine, you can look for
a hosting provider who offers a Linux \emph{shell account}, which
is a command-line only interface to a Linux machine.  
There are many low-cost
shell account providers, but you have to make sure that they match the 
requirements above (i.e. - Linux on x86).  
Someone at your local GNU/Linux User's Group may be able to give you one as 
well.  Shell accounts only require
that you already have an Internet connection and a telnet program.  If you use
Windows\textregistered, you already have a telnet client - just click on 
\icode{start}, then \icode{run}, then type in 
\icode{telnet}.  However, it is usually better to download
\documentname{PuTTY} from
http://www.chiart.greenend.co.uk/~sgtatham/putty/
because Windows' telnet has some weird problems.  There are a lot of options
for the Macintosh, too.  \documentname{NiftyTelnet} is my 
favorite.

If you don't have GNU/Linux\index{GNU/Linux}
and can't find a shell account service, then you
can download \documentname{Knoppix} from http://www.knoppix.org/
Knoppix\index{Knoppix} is a 
GNU/Linux distribution that boots from CD so that you don't have
to actually install it.  Once you are done using it, you just reboot and
remove the CD and you are back to your regular operating system.

So what is GNU/Linux?  GNU/Linux\index{GNU/Linux} is an operating system modeled after
UNIX\textregistered.  The GNU part comes from the \url{GNU 
Project}\footnote{The GNU Project is a project by the Free
Software Foundation to produce a complete, free operating 
system.}, which includes most of the programs you 
will run, including
the GCC\index{GCC} tool set that we 
will use to program with.  The GCC tool set
contains all of the programs necessary to create programs in various
computer languages.

Linux\index{Linux} is the name
of the \emph{kernel}.  The kernel\index{kernel} is the core part of an
operating system that keeps track of everything.  The kernel is
both a fence and a gate.  As a gate, it allows programs
to access hardware in a uniform way.  Without the kernel, you would have
to write programs to deal with every device model ever made.  The kernel
handles all device-specific interactions so you don't have to.  It also handles
file access and interaction between processes.  
For example, when you
type, your typing goes through several programs before it hits your editor.
First, the kernel is what handles your hardware, so it is the first to receive
notice about the keypress.  The keyboard sends in 
\emph{scancodes} to the kernel, which then converts them to the
actual letters, numbers, and symbols they represent.  If you are using a
windowing system (like Microsoft Windows\textregistered or the X Window System), then the
windowing system reads the keypress from the kernel, and delivers it to
whatever program is currently in focus on the user's display.

\begin{figure}

\begin{simpletyping}
\begin{lstlisting}
Keyboard -> Kernel -> Windowing system -> Application program
\end{lstlisting}
\end{simpletyping}

\caption{How the computer processes keyboard sigals}
\end{figure}

The kernel\index{kernel} also controls the flow of information between programs.  
The kernel is a program's gate to the world around it.  Every time that
data moves between processes, the kernel controls the messaging.  In our
keyboard example above, the kernel would have to be involved for the
windowing system to communicate the keypress to the application program.

As a fence, the kernel
prevents programs from accidentally overwriting each other's data and from
accessing files and devices that they don't have permission to.  It
limits the amount of damage a poorly-written program can do to other 
running programs.

In our case, the kernel\index{kernel}
is Linux\index{Linux}.  Now, the kernel 
all by itself won't do anything.  You can't even
boot up a computer with just a kernel.  Think of the kernel as the water pipes
for a house.  Without the pipes, the faucets won't work, but the pipes are
pretty useless if there are no faucets.  Together, the user applications
(from the GNU project and other places) and the kernel (Linux) make 
up the entire operating system, GNU/Linux.

For the most part, this book will be using the computer's low-level
assembly language.  There are essentially three kinds of languages:

\index{machine language}

\begin{description}
\item[Machine Language] This is what the computer actually sees and deals with.  Every
command the computer sees is given as a number or sequence of
numbers.
\item[Assembly Language] This is the same as machine language, except the command numbers
have been replaced by letter sequences which are easier to memorize.
Other small things are done to make it easier as well.
\item[High-Level Language\index{high-level languages}] High-level languages are there to make programming easier.  Assembly
language requires you to work with the machine itself.  High-level
languages allow you to describe the program in a more natural language.
A single command in a high-level language usually is equivalent to
several commands in an assembly language.
\end{description}

\index{Assembly Language}
In this book we will learn assembly language, although we will cover a
bit of high-level languages.  
Hopefully by learning assembly language, your understanding of how programming
and computers work will put you a step ahead.

% 
% 

% 
% 

% Assembly language has long fallen out of vogue in computer science education, and I believe
% that this is causing the current generation of programmers to have a much
% weaker foundation.   Assembly language was being viewed as an optional add-on 
% that was only useful for operating-system programmers.  I wrote this book
% to bring assembly language to the forefront of programmer education.  I believe this
% is important, because while interviewing programmers for jobs, I have found
% that the near-universal trait of qualified candidates was that they all knew
% assembly language.  Few were experts and even fewer used it regularly, but
% everyone who knew assembly language had a deeper understanding of programming
% issues than the other candidates.
% 

% 
% 

% 

% 

