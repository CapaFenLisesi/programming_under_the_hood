\chapter{High-Level Languages}
\label{highlevellanguages}

% 
% 
% Copyright 2002 Jonathan Bartlett
% 
% Permission is granted to copy, distribute and/or modify this
% document under the terms of the GNU Free Documentation License,
% Version 1.1 or any later version published by the Free Software
% Foundation; with no Invariant Sections, with no Front-Cover Texts,
% and with no Back-Cover Texts.  A copy of the license is included in fdl.xml
% 

In this chapter we will begin to look at our first "real-world"
programming language.  Assembly language is the language used
at the machine's level, but most people find
coding in assembly language too cumbersome for everyday use.
Many computer languages have been invented to make the programming
task easier.  Knowing a wide variety of languages is useful for
many reasons, including

\begin{itemize}\item Different languages are based on different concepts, which will help you to learn different and better programming methods and ideas. 
\item Different languages are good for different types of projects. 
\item Different companies have different standard languages, so knowing more languages makes your skills more marketable. 
\item The more languages you know, the easier it is to pick up new ones. 
\end{itemize}

As a programmer, you will often have to pick up new languages.  Professional
programmers can usually pick up a new language with about a weeks
worth of study and practice.  Languages are simply tools, and learning
to use a new tool should not be something a programmer flinches at.  In fact,
if you do computer consulting you will often have to learn new languages
on the spot in order to keep yourself employed. It will often be your
customer, not you, who decides what language is used.  This
chapter will introduce you to a few of the languages available to you.  
I encourage you to explore as many languages as you are interested in.
I personally try to learn a new language every few months.

\section{Compiled and Interpreted Languages}

Many languages are \emph{compiled} languages.  
When you write assembly language\index{assembly language}, each instruction you write is 
translated into
exactly one machine instruction for processing.  With compilers\index{compilers},
a statement can translate into one or hundreds of machine 
instructions.  In fact, depending on how advanced your compiler
is, it might even restructure parts of your code to make it faster.
In assembly language what you write is what you get.

There are also languages that are \emph{interpreted}
languages.  These languages require that the user run a program
called an \emph{interpreter\index{interpreter}} that in turn runs the
given program.  These are usually slower than compiled programs,
since the interpreter has to read and interpret the code as it goes 
along.  However, in well-made interpreters, this time can be
fairly negligible.  There is also a class of hybrid languages
which partially compile a program before execution into byte-codes.
This is done because the interpreter
can read the byte-codes much faster than it can read the regular language.

There are many reasons to choose one or the other.  Compiled programs
are nice, because you don't have to already have an interpreter installed
in the user's machine.  You have to have a compiler for the language,
but the users of your program don't.  In an interpreted language, you
have to be sure that the user has an interpreter installed for your program,
and that the computer knows which interpreter to run your program with.  However,
interpeted languages tend to be more flexible, while compiled languages
are more rigid.

Language choice is usually driven by available tools and support for 
programming methods rather than by whether a language is compiled
or interpretted.  In fact many languages have options for either one.  

High-level languages, whether compiled or interpreted, are oriented around
you, the programmer, instead of around the machine.  This opens them up
to a wide variety of features, which can include the following:

\begin{itemize}\item Being able to group multiple operations into a single expression 
\item Being able to use "big values" - values that are much more conceptual than the 4-byte words that computers normally deal with (for example, being able to view text strings as a single value rather than as a string of bytes). 
\item Having access to better flow control\index{flow control} constructs than just jumps. 
\item Having a compiler to check types of value assignments and other assertions. 
\item Having memory handled automatically. 
\item Being able to work in a language that resembles the problem domain rather than the computer hardware. 
\end{itemize}

So why does one choose one language over another?
For example, many choose Perl because it has a vast library of functions
for handling just about every protocol or type of data on the planet.
Python, however, has a cleaner syntax and often lends itself to more 
straightforward solutions.  Its cross-platform GUI tools are also excellent.  
PHP makes writing web applications simple.  Common LISP has more power and
features than any other environment for those willing to learn it.  Scheme is
the model of simplicity and power combined together.  C is easy to interface
with other languages.

Each language is different, and the more 
languages you know the better programmer you will be.  Knowing the
concepts of different languages will help you in all programming, because
you can match the programming language to the problem better, and you
have a larger set of tools to work with.  Even if certain features aren't
directly supported in the language you are using, often they can be simulated.
However, if you don't have a broad experience with languages, you won't
know of all the possibilities you have to choose from.

\section{Your First C Program}

\index{C Programming Language}
Here is your first C program,
which prints "Hello world" to the screen and exits.  Type it in, and
give it the name Hello-World.c

\begin{simpletyping}
\lstinputlisting{Hello-World-c}
\end{simpletyping}

As you can see, it's a pretty simple program.  To compile it,
run the command

\begin{simpletyping}
\begin{lstlisting}
gcc -o HelloWorld Hello-World.c
\end{lstlisting}
\end{simpletyping}

To run the program, do

\begin{simpletyping}
\begin{lstlisting}
./HelloWorld
\end{lstlisting}
\end{simpletyping}

Let's look at how this program was put together.

Comments in C are started with \icode{/*} and ended with
\icode{*/}.  Comments can span multiple lines, but 
many people prefer to start and end comments on the same line
so they don't get confused.

\icode{\#include <stdio.h>}
\index{stdio.h}
 is the first part of
the program.  This is a \emph{preprocessor directive}.
C compiling is split into two stages - the preprocessor\index{preprocessor} and the
main compiler.  This directive tells the preprocessor to look
for the file \icodefilename{stdio.h} and paste it into
your program.  The preprocessor is responsible for putting together the
text of the program.  This includes sticking different files together, running
macros on your program text, etc.  After the text is put together, the 
preprocessor is done and the main compiler goes to work.

Now, everything in \icodefilename{stdio.h} is
now in your program just as if you typed it there yourself.  The angle brackets
around the filename tell the compiler to look in its standard
paths for the file (\icodefilename{/usr/include\index{/usr/include}} and 
\icodefilename{/usr/local/include\index{/usr/local/include}}, usually).  If it was
in quotes, like \icode{\#include "stdio.h"} it would
look in the current directory for the file.  Anyway, 
\icodefilename{stdio.h} contains the declarations for 
the standard input and output functions and variables.   These
declarations tell the
compiler what functions are available for input and output.  The next few
lines are simply comments about the program.

Then there is the line \icode{int main(int argc, char **argv)}.
This is the start of a function.  C Functions are declared with their 
name, arguments and return type.  This declaration says that the function's
name is \icode{main}, it returns an \icode{int} 
(integer - 4 bytes long on the x86 platform), and has two arguments - 
an \icode{int} called
\icode{argc} and a \icode{char **} called 
\icode{argv}.  You don't have to worry about where the arguments
are positioned on the stack - the C compiler takes care of that for you.
You also don't have to worry about loading values into and out of registers
because the compiler takes care of that, too.

The \icode{main} function is a special function in the
C language - it is the
start of all C programs (much like \icode{\_start} in our
assembly-language programs).  It always takes two parameters.  The first 
parameter is the number of arguments given to this command, and the
second parameter is a list of the arguments that were given.

The next line is a function call.  In assembly language, you had to 
push the arguments of a function onto the stack, and then call the function.
C takes care of this complexity for you.  You simply have to call the
function with the parameters in parenthesis.  In this case, we call
the function \icode{puts}, with a single parameter.  This
parameter is the character string we want to print.  We just have to
type in the string in quotations, and the compiler takes care of defining
storage and moving the pointers to that storage onto the stack before
calling the function.  As you can see, it's a lot less work.

Finally our function returns the number \icode{0}.  In assembly
language, we stored our return value in {\eaxReg}, but in C we just use
the \icode{return} command and it takes care of that for us.
The return value of the \icode{main} function is what is
used as the exit code for the program.

As you can see, using high-level languages\index{high-level languages} makes life much easier.
It also allows our programs to run on multiple platforms more easily.
In assembly language, your program is tied to both the operating system
and the hardware platform, while in compiled and interpreted languages
the same code can usually run on multiple operating systems and hardware
platforms.  For example, this program can be built and executed on x86
hardware running Linux\textregistered, Windows\textregistered, UNIX\textregistered, or most other operating systems.
In addition, it can also run on Macintosh hardware running a number of
operating systems.

Additional information on the C programming language can be found in
\autoref{ctranslationap}.

\section{Perl}

Perl is an interpreted language, existing mostly on Linux and UNIX-based
platforms.  It actually runs on almost all platforms, but you find it most
often on Linux and UNIX-based ones.  Anyway, here is the Perl version
of the program, which should be typed into a file named 
\icodefilename{Hello-World.pl}:

\begin{simpletyping}
\lstinputlisting{Hello-World-perl}
\end{simpletyping}

Since Perl is interpreted, you don't need to compile or link it.  Just
run in with the following command:

\begin{simpletyping}
\begin{lstlisting}
perl Hello-World.pl
\end{lstlisting}
\end{simpletyping}

As you can see, the Perl\index{Perl} version is even shorter than the C version.
With Perl you don't have to declare any functions or program entry points.
You can just start typing commands and the interpreter will run them
as it comes to them.  In fact this program only has two lines of code,
one of which is optional.

The first, optional line is used for UNIX machines to tell which interpreter
to use to run the program.  The \icode{\#!} tells the computer
that this is an interpreted program, and the 
\icodefilename{/usr/bin/perl} tells the computer to use the program
\icodefilename{/usr/bin/perl} to interpret the program.  However, since
we ran the program by typing in \icode{perl Hello-World.pl}, we
had already specified that we were using the perl interpreter.

The next line calls a Perl builtin function, print.  This has one parameter,
the string to print.  The program doesn't have an explicit return
statement - it knows to return simply because it runs off the end of the file.
It also knows to return 0 because there were no errors while it ran.  You can
see that interpreted languages are often focused on letting you get working 
code as quickly as possible, without having to do a lot of extra legwork.

One thing about Perl that isn't so evident from this example is that Perl
treats strings as a single value.  In assembly language, we had to program
according to the computer's memory architecture, which meant that strings
had to be treated as a sequence of multiple values, with a pointer to the
first letter.  Perl pretends that strings can be stored directly as values,
and thus hides the complication of manipulating them for you.  In fact,
one of Perl's main strengths is its ability and speed at manipulating
text.  

\section{Python}

The Python\index{Python} version of the program looks almost exactly like the Perl one.
However, Python is really a very different language than Perl, even if
it doesn't seem so from this trivial example.  Type the program into
a file named \icodefilename{Hello-World.py}.  The program follows:

\begin{simpletyping}
\lstinputlisting{Hello-World-python}
\end{simpletyping}

You should be able to tell what the different lines of the program do.

\section{Review}

\section{Know the Concepts}

\begin{itemize}\item What is the difference between an intepretted language and a compiled language? 
\item What reasons might cause you to need to learn a new programming language? 
\end{itemize}

\section{Use the Concepts}

\begin{itemize}\item Learn the basic syntax of a new programming language.  Re-code one of the programs in this book in that language. 
\item In the program you wrote in the question above, what specific things were automated in the programming language you chose? 
\item Modify your program so that it runs 10,000 times in a row, both in assembly language and in your new language.  Then run the \icode{time} command to see which is faster.  Which does come out ahead?  Why do you think that is? 
\item How does the programming language's input/output methods differ from that of the Linux system calls? 
\end{itemize}

\section{Going Further}

\begin{itemize}\item Having seen languages which have such brevity as Perl, why do you think this book started you with a language as verbose as assembly language? 
\item How do you think high level languages have affected the process of programming? 
\item Why do you think so many languages exist? 
\item Learn two new high level languages.  How do they differ from each other?  How are they similar?  What approach to problem-solving does each take? 
\end{itemize}

% 
% 
% Advanced Linking Techniques
% 
% LD\_PRELOAD
% 
% loading a library on the fly
% 
% static libraries
% 
% sonames
% 
% ELF sections and special sections (.ini .fini or whatever)
% 
% Process for loading and linking ELF files
% 
% Other object file types (COFF a.out)
% 
% The file format used to store programs in Linux is called ELF, which
% stands for "Executable and Linking Format".  ELF is very flexible,
% but it is also fairly complex.  Let's take a look at how we've
% been using ELF so far.
% 

% 
% <sect2>
% \textbf{ELF Sections}
% 

% The \icode{.section} directives we've been using
% are commands that tell the linker, \icode{ld}, to
% group things together.  There are many predefined sections.
% The ones we mainly use are \icode{.data} and 
% \icode{.text}.  In your assembly code, you can mix
% \icode{.data} and \icode{.text} sections,
% and the linker will put everything in a section together.  Let's
% take a look at one of our files.  Let's look at the 
% \icode{toupper} file we created earlier.  If you
% run the command \icode{objdump -h toupper}, you will
% get a listing that looks like this:
% 

% 
% 

